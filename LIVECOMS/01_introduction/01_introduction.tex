%Author: Lester Hedges Email:~~ lester.hedges@bristol.ac.uk

\hypertarget{biosimspace}{%
\section{BioSimSpace}\label{biosimspace}}

The companion notebook for this section can be found
\href{https://github.com/michellab/BioSimSpaceTutorials/blob/4844562e7d2cd0b269cead56562ec16a3dfaef7c/01_introduction/01_introduction.ipynb}{here}.

\hypertarget{introduction}{%
\subsection{Introduction}\label{introduction}}

Welcome to this workshop on \href{https://biosimspace.org}{BioSimSpace},
an \emph{interoperable} Python framework for biomolecular simulation. In
this introductory session you will learn:

\begin{itemize}
\tightlist
\item
  What are the key concepts behind BioSimSpace.
\item
  How to set up molecular systems ready for simulation.
\item
  How to configure and run a range of molecular dynamics protocols using
  different simulation engines.
\item
  How to write interoperable workflow components and run them in a
  variety of ways.
\end{itemize}

\hypertarget{what-is-biosimspace}{%
\subsection{What is BioSimSpace?}\label{what-is-biosimspace}}

As a computational chemist, you are likely overwhelmed by the amount of
different software packages that are available to you. Having choice is
a good thing, but too much can become a burden. I'm sure you have all
come across at least one of the following:

\begin{itemize}
\tightlist
\item
  I know how to solve the problem with package X but I want to use
  package Y.
\item
  How can I share my script with a collaborator who doesn't use the same
  software stack?
\item
  How can I take advantage of the best tool for the job for different
  parts of my workflow?
\item
  How can I compare methodology / results between simulation engines?
\end{itemize}

Solving these problems is the core goal of BioSimSpace. The wealth of
fantastic software in our community is a real asset but
\emph{interoperability} is currently a problem. Since there is no point
reinventing the wheel, BioSimSpace is not an attempt to produce yet
another molecular simulation package that reproduces all of the
functionality from existing programs. This would result in just another
tool for you to learn, along with yet another set of standards and
formats. Instead, BioSimSpace is essentially just a set of \emph{shims},
or bits of \emph{glue}, that connect together existing software
packages, allowing you to interact with them using a consistent Python
interface.

\hypertarget{why-biosimspace}{%
\subsection{Why BioSimSpace?}\label{why-biosimspace}}

By using BioSimSpace you will be less reliant on the use of brittle
scripts to connect different software packages together. BioSimSpace
builds on top of existing and open Python tools within the biomolecular
community, e.g. \href{https://www.rdkit.org/}{RDKit},
\href{http://openmm.org/}{OpenMM},
\href{https://github.com/openforcefield/openff-toolkit}{Open Force
Field}. As such, you are able to leverage the power of other packages,
with which you may already be familiar, and to mix-and-match
functionality where required.

With BioSimSpace you will be able to:

\begin{itemize}
\tightlist
\item
  Write generic workflow components \emph{once} in a package-agnostic
  language.
\item
  Run the same script from the command-line, Jupyter, or within a
  workflow engine.
\item
  Use the most suitable package that is availabe on your computer.
\item
  Continue using your favourite package X but be able to share scripts
  with your collaborator who prefers package Y.
\item
  Be able to take advantage of new software packages and hardware
  resources as and when they become available.
\end{itemize}

\hypertarget{what-can-biosimspace-do}{%
\subsection{What can BioSimSpace do?}\label{what-can-biosimspace-do}}

BioSimSpace provides a suite of packages with a range of different
functionality.

At present:

\begin{itemize}
\tightlist
\item
  File conversion:
  \href{https://biosimspace.org/api/index_IO.html}{BioSimSpace.IO}
\item
  Parameterisation:
  \href{https://biosimspace.org/api/index_Parameters.html}{BioSimSpace.Parameters}
\item
  Solvation:
  \href{https://biosimspace.org/api/index_Solvent.html}{BioSimSpace.Solvent}
\item
  Molecular dynamics:
  \href{https://biosimspace.org/api/index_Protocol.html}{BioSimSpace.Protocol},
  \href{https://biosimspace.org/api/index_Process.html}{BioSimSpace.Process},
  \href{https://biosimspace.org/api/index_MD.html}{BioSimSpace.MD}
\item
  Free-energy perturbation:
  \href{https://biosimspace.org/api/index_Align.html}{BioSimSpace.Align},
  \href{https://biosimspace.org/api/index_FreeEnergy.html}{BioSimSpace.FreeEnergy}
\item
  Metadynamics:
  \href{https://biosimspace.org/api/index_Metadynamics.html}{BioSimSpace.Metadynamics}
\item
  Trajectory handling:
  \href{https://biosimspace.org/api/index_Trajectory.html}{BioSimSpace.Trajectory}
\item
  Interactive visualisation:
  \href{https://biosimspace.org/api/index_Notebook.html}{BioSimSpace.Notebook}
\item
  Workflow components:
  \href{https://biosimspace.org/api/index_Gateway.html}{BioSimSpace.Gateway}
\end{itemize}

\hypertarget{key-concepts}{%
\subsection{Key concepts}\label{key-concepts}}

Before getting started it's worth spending a little time covering a few
of the key concepts of BioSimSpace.

\hypertarget{file-conversion}{%
\subsubsection{File conversion}\label{file-conversion}}

While, broadly speaking, the different molecular dynamics engines offer
a similar range of features, their interfaces are quite different. This
makes it hard to take expertise in one package and immediately apply it
to another. At the heart of this problem is the incompatibility between
the molecular file formats used by the different packages. While they
all contain the same information, i.e.~how atoms are laid out in space
and how they interact with each other, the structure of the files is
very different. In order to provide interoperability betwen packages we
will need to be able to read and write many different file formats, and
be able to interconvert between them too.

Let's import the BioSimSpace Python package and see what we can do. For
convenience, we'll rename the package to BSS to save us typing:

\begin{Shaded}
\begin{Highlighting}[]
\ImportTok{import}\NormalTok{ BioSimSpace }\ImportTok{as}\NormalTok{ BSS}
\end{Highlighting}
\end{Shaded}

To see what file formats are supported by BioSimSpace, execute the cell
below.

\begin{Shaded}
\begin{Highlighting}[]
\NormalTok{BSS.IO.fileFormats()}
\end{Highlighting}
\end{Shaded}

\begin{verbatim}
['Gro87', 'GroTop', 'MOL2', 'PDB', 'PRM7', 'PSF', 'RST', 'RST7']
\end{verbatim}

Note that these refer to specific file \emph{formats}, rather than file
\emph{extensions}. BioSimSpace doesn't care about file extensions, it's
the \emph{contents} of the file that's important.

If you aren't familiar with a particular format, you can get more
information as follows, e.g.:

\begin{Shaded}
\begin{Highlighting}[]
\NormalTok{BSS.IO.formatInfo(}\StringTok{"GroTop"}\NormalTok{)}
\end{Highlighting}
\end{Shaded}

\begin{verbatim}
'Gromacs Topology format files.'
\end{verbatim}

The \texttt{BSS.IO.readMolecules} function is used to read molecular
information from file. We've provided some example input files for you
in the \texttt{inputs} directory. Let's take a look at some of these.

\begin{Shaded}
\begin{Highlighting}[]
\OperatorTok{!}\NormalTok{ls inputs}
\end{Highlighting}
\end{Shaded}

\begin{verbatim}
1jr5.crd  1jr5.top  ala.crd  kigaki.gro  methanol.pdb
1jr5.pdb  2JJC.pdb  ala.top  kigaki.top
\end{verbatim}

The \texttt{ala.crd} and \texttt{ala.top} files define a solvated
alanine dipeptide system in AMBER format. Execute the cell below to see
part of the topology file:

\begin{Shaded}
\begin{Highlighting}[]
\OperatorTok{!}\NormalTok{head }\OperatorTok{-}\NormalTok{n }\DecValTok{20}\NormalTok{ inputs}\OperatorTok{/}\NormalTok{ala.top}
\end{Highlighting}
\end{Shaded}

\begin{verbatim}
%VERSION  VERSION_STAMP = V0001.000  DATE = 06/30/15  11:44:23                  
%FLAG TITLE                                                                     
%FORMAT(20a4)                                                                   
ACE                                                                             
%FLAG POINTERS                                                                  
%FORMAT(10I8)                                                                   
    1912       9    1902       9      25      11      43      24       0       0
    2619     633       9      11      24      13      21      20      10       1
       0       0       0       0       0       0       0       1      10       0
       0
%FLAG ATOM_NAME                                                                 
%FORMAT(20a4)                                                                   
HH31CH3 HH32HH33C   O   N   H   CA  HA  CB  HB1 HB2 HB3 C   O   N   H   CH3 HH31
HH32HH33O   H1  H2  O   H1  H2  O   H1  H2  O   H1  H2  O   H1  H2  O   H1  H2  
O   H1  H2  O   H1  H2  O   H1  H2  O   H1  H2  O   H1  H2  O   H1  H2  O   H1  
H2  O   H1  H2  O   H1  H2  O   H1  H2  O   H1  H2  O   H1  H2  O   H1  H2  O   
H1  H2  O   H1  H2  O   H1  H2  O   H1  H2  O   H1  H2  O   H1  H2  O   H1  H2  
O   H1  H2  O   H1  H2  O   H1  H2  O   H1  H2  O   H1  H2  O   H1  H2  O   H1  
H2  O   H1  H2  O   H1  H2  O   H1  H2  O   H1  H2  O   H1  H2  O   H1  H2  O   
H1  H2  O   H1  H2  O   H1  H2  O   H1  H2  O   H1  H2  O   H1  H2  O   H1  H2  
\end{verbatim}

Let's now read the molecules from file. The
\texttt{BSS.IO.readMolecules} function automatically
\href{https://en.wikipedia.org/wiki/Glob_(programming)}{globs} the
passed string, so wildcard matching can be used to determine the files.
(Here the asterisk matches any characters, i.e.~we are reading
\emph{all} files in the \texttt{inputs} directory with \texttt{ala} as
the file prefix.)

\begin{Shaded}
\begin{Highlighting}[]
\NormalTok{system }\OperatorTok{=}\NormalTok{ BSS.IO.readMolecules(}\StringTok{"inputs/ala.*"}\NormalTok{)}
\end{Highlighting}
\end{Shaded}

N.B. We could have explictly specified each file using a list of
strings.

Note that we don't have to specify anything about the file format.
BioSimSpace actually reads the files with \emph{all} of its inbuilt
parsers in parallel. If a parser fails to read the file then it is
immediately rejected and we move on to the next. As long as all of the
files that are read contain a consistent topology, then BioSimSpace will
be able to read them.

To see what file formats are associated with the system, run:

\begin{Shaded}
\begin{Highlighting}[]
\NormalTok{system.fileFormat()}
\end{Highlighting}
\end{Shaded}

\begin{verbatim}
'PRM7,RST7'
\end{verbatim}

As expected, BioSimpace has detected that these were AMBER format
topology and coordinate files.

The molecules are now loaded into a \texttt{System} object. BioSimSpace
objects are thin wrappers around the equivalent objects from
\href{https://github.com/michellab/Sire}{Sire}. For those that are
familiar with Sire, you can get always access the underlying object
directly using \texttt{system.\_sire\_object}. This \emph{private}
member is hidden from the user. Sire provides an extremely powerful and
flexible set of tools for molecular manipulation and editing. While
BioSimSpace directly exposes only a small subset of this functionality
to the user, the full power of Sire is always easily available when
needed.

We can query how many molecules there are as follows:

\begin{Shaded}
\begin{Highlighting}[]
\NormalTok{system.nMolecules()}
\end{Highlighting}
\end{Shaded}

\begin{verbatim}
631
\end{verbatim}

To see how many water molecules there are:

\begin{Shaded}
\begin{Highlighting}[]
\NormalTok{system.nWaterMolecules()}
\end{Highlighting}
\end{Shaded}

\begin{verbatim}
630
\end{verbatim}

To search for all nitrogen atoms in residues named \texttt{ALA}:

\begin{Shaded}
\begin{Highlighting}[]
\NormalTok{system.search(}\StringTok{"element N and resname ALA"}\NormalTok{)}
\end{Highlighting}
\end{Shaded}

\begin{verbatim}
<BioSimSpace.SearchResult: nResults=1>
\end{verbatim}

Create a new system using from the first 10 and last molecule in the
system:

\begin{Shaded}
\begin{Highlighting}[]
\NormalTok{new_system }\OperatorTok{=}\NormalTok{ (system[:}\DecValTok{10}\NormalTok{] }\OperatorTok{+}\NormalTok{ system[}\OperatorTok{-}\DecValTok{1}\NormalTok{]).toSystem()}
\end{Highlighting}
\end{Shaded}

N.B. Information from the topology file pertaining to the molecular
force field are stored internally as computer algebra expressions,
allowing us to mathematically interconvert between different
representations of the terms when writing to a different format. If
desired, you could even edit the system to create your own unique force
field parameters, although this beyond the scope of this tutorial.

Now that we have a molecular system, let's write it back to disk in a
different format. Unsurprisingly, this is done using the
\texttt{BSS.IO.saveMolecules} function. Execute the cell below to write
the system to GROMACS format coordinate and topology files.

\begin{Shaded}
\begin{Highlighting}[]
\NormalTok{BSS.IO.saveMolecules(}\StringTok{"ala"}\NormalTok{, system, [}\StringTok{"Gro87"}\NormalTok{, }\StringTok{"GroTop"}\NormalTok{])}
\end{Highlighting}
\end{Shaded}

\begin{verbatim}
['/home/lester/Code/BioSimSpaceTutorials/01_introduction/ala.gro',
 '/home/lester/Code/BioSimSpaceTutorials/01_introduction/ala.top']
\end{verbatim}

Run the cell below to examine the start of the GROMACS topology file.

\begin{Shaded}
\begin{Highlighting}[]
\OperatorTok{!}\NormalTok{head }\OperatorTok{-}\NormalTok{n }\DecValTok{20}\NormalTok{ ala.top}
\end{Highlighting}
\end{Shaded}

\begin{verbatim}
; Gromacs Topology File written by Sire
; File written 05/06/21  14:17:35
[ defaults ]
; nbfunc      comb-rule       gen-pairs      fudgeLJ     fudgeQQ
  1           2               yes            0.5         0.833333

[ atomtypes ]
; name      at.num        mass      charge   ptype       sigma     epsilon
     C           6   12.010700    0.000000       A    0.339967    0.359824
    CT           6   12.010700    0.000000       A    0.339967    0.457730
    CX           6   12.010700    0.000000       A    0.339967    0.457730
     H           1    1.007940    0.000000       A    0.106908    0.065689
    H1           1    1.007940    0.000000       A    0.247135    0.065689
    HC           1    1.007940    0.000000       A    0.264953    0.065689
    HW           1    1.007940    0.000000       A    0.000000    0.000000
     N           7   14.006700    0.000000       A    0.325000    0.711280
     O           8   15.999400    0.000000       A    0.295992    0.878640
    OW           8   15.999400    0.000000       A    0.315061    0.636386

[ moleculetype ]
\end{verbatim}

\hypertarget{topology-preservation}{%
\subsubsection{Topology preservation}\label{topology-preservation}}

One of the other pitfalls of working with different molecular simulation
engines is that they often have quirks regarding naming conventions and
atom ordering. This means that what you get back from a given program
might not be the same as what you put in, making it tricky to
cross-reference parts of the system that are of specific interest.

N.B. Differerent naming conventions are one of the hardest problems with
interoperability.

BioSimSpace tries to preserve the intial molecular topology during any
interaction with external tools so that it can be used as a consistent
reference. For example, while we might need to rename the water topology
to match the conventions of a particular molecular dynamics engine, we
always copy the updated coordinates from a simulation back into the
original system so that the naming that the user gets back is unchanged.

Another common topology feature that can be lost is chain identifiers.
For example, a \href{https://www.rcsb.org/}{Protein Data Bank} file
might contain labels for chains, but these would be lost during
parameterisation with the
\href{https://ambermd.org/AmberTools.php}{AmberTools} suite since it has
no internal concept of chains.

As an example, consider the following PDB file:

\begin{Shaded}
\begin{Highlighting}[]
\NormalTok{system_pdb }\OperatorTok{=}\NormalTok{ BSS.IO.readMolecules(}\StringTok{"inputs/1jr5.pdb"}\NormalTok{)}
\end{Highlighting}
\end{Shaded}

This is read as a single molecule containing two chains:

\begin{Shaded}
\begin{Highlighting}[]
\BuiltInTok{print}\NormalTok{(}\SpecialStringTok{f"mols = }\SpecialCharTok{\{}\NormalTok{system_pdb}\SpecialCharTok{.}\NormalTok{nMolecules()}\SpecialCharTok{\}}\SpecialStringTok{, chains = }\SpecialCharTok{\{}\NormalTok{system_pdb}\SpecialCharTok{.}\NormalTok{nChains()}\SpecialCharTok{\}}\SpecialStringTok{"}\NormalTok{)}
\end{Highlighting}
\end{Shaded}

\begin{verbatim}
mols = 1, chains = 2
\end{verbatim}

The molecule in the system has a topology but no force field
information, as we can see by querying the \emph{properties} of the
underlying Sire object:

\begin{Shaded}
\begin{Highlighting}[]
\NormalTok{system_pdb[}\DecValTok{0}\NormalTok{]._sire_object.propertyKeys()}
\end{Highlighting}
\end{Shaded}

\begin{verbatim}
['alt_loc',
 'beta_factor',
 'coordinates',
 'element',
 'insert_code',
 'formal_charge',
 'fileformat',
 'occupancy',
 'is_het']
\end{verbatim}

After parameterising the molecule with an AMBER protein force field, we
end up with the following system:

\begin{Shaded}
\begin{Highlighting}[]
\NormalTok{system_amber }\OperatorTok{=}\NormalTok{ BSS.IO.readMolecules([}\StringTok{"inputs/1jr5.crd"}\NormalTok{, }\StringTok{"inputs/1jr5.top"}\NormalTok{])}
\end{Highlighting}
\end{Shaded}

In contrast, this is read as a two molecules with zero chains:

\begin{Shaded}
\begin{Highlighting}[]
\BuiltInTok{print}\NormalTok{(}\SpecialStringTok{f"mols = }\SpecialCharTok{\{}\NormalTok{system_amber}\SpecialCharTok{.}\NormalTok{nMolecules()}\SpecialCharTok{\}}\SpecialStringTok{, chains = }\SpecialCharTok{\{}\NormalTok{system_amber}\SpecialCharTok{.}\NormalTok{nChains()}\SpecialCharTok{\}}\SpecialStringTok{"}\NormalTok{)}
\end{Highlighting}
\end{Shaded}

\begin{verbatim}
mols = 2, chains = 0
\end{verbatim}

Molecules in the parameterised system contain a different set of
properties, including those pertaining to the force field.

\begin{Shaded}
\begin{Highlighting}[]
\NormalTok{system_amber[}\DecValTok{0}\NormalTok{]._sire_object.propertyKeys()}
\end{Highlighting}
\end{Shaded}

\begin{verbatim}
['intrascale',
 'bond',
 'angle',
 'fileformat',
 'atomtype',
 'LJ',
 'charge',
 'connectivity',
 'forcefield',
 'mass',
 'coordinates',
 'treechain',
 'gb_radii',
 'ambertype',
 'improper',
 'element',
 'parameters',
 'gb_screening',
 'gb_radius_set',
 'dihedral']
\end{verbatim}

If we want to preserve the topology of the original molecule, yet add
the updated properties from the molecules in the new system, then we can
make it \emph{compatible}. BioSimSpace provides functionality to do this
for you:

\begin{Shaded}
\begin{Highlighting}[]
\CommentTok{# Extract the original molecule.}
\NormalTok{new_mol }\OperatorTok{=}\NormalTok{ system_pdb[}\DecValTok{0}\NormalTok{]}

\CommentTok{# Make it compatible with the new system, i.e. add new properties while}
\CommentTok{# preserving the topology.}
\NormalTok{new_mol.makeCompatibleWith(system_amber)}
\end{Highlighting}
\end{Shaded}

N.B. This is done implicitly whenever calling any BioSimSpace
parameterisation function.

Let's now check the number of chains in the new molecule, as well as the
properties that are associated with it.

\begin{Shaded}
\begin{Highlighting}[]
\BuiltInTok{print}\NormalTok{(}\SpecialStringTok{f"chains = }\SpecialCharTok{\{}\NormalTok{new_mol}\SpecialCharTok{.}\NormalTok{nChains()}\SpecialCharTok{\}}\SpecialStringTok{"}\NormalTok{)}
\NormalTok{new_mol._sire_object.propertyKeys()}
\end{Highlighting}
\end{Shaded}

\begin{verbatim}
chains = 2





['formal_charge',
 'is_het',
 'intrascale',
 'bond',
 'angle',
 'fileformat',
 'atomtype',
 'LJ',
 'charge',
 'connectivity',
 'forcefield',
 'alt_loc',
 'mass',
 'beta_factor',
 'coordinates',
 'occupancy',
 'treechain',
 'gb_radii',
 'ambertype',
 'improper',
 'element',
 'insert_code',
 'gb_screening',
 'gb_radius_set',
 'dihedral']
\end{verbatim}
